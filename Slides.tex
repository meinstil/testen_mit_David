\documentclass[large]{beamer}

\mode<presentation>
%\mode<handout>
{
  \usetheme{Malmoe}
  \usecolortheme{default}
  
  % Malmoe WITH default / seagull / [named=COLOR]{structure}
  % Dresden WITH default / dove / 
}

\usepackage[english]{babel}
\usepackage[latin1]{inputenc}
\usepackage{graphicx}
\usepackage{theorem}
\usepackage{multirow}
\usepackage{amsmath}
\usepackage{amssymb}
\usepackage{epsfig}
\usepackage{colordvi}
\usepackage{xcolor}

%\usepackage{tabularx}
%\usepackage{booktabs}
%\usepackage{dcolumn}
%\usepackage{paralist}
%\setbeamertemplate{itemize item}[default]

%\usepackage[T1]{fontenc}
% Or whatever. Note that the encoding and the font should match. If T1
% does not look nice, try deleting the line with the fontenc.


\title[SHORT TITLE]{TITLE}

%\subtitle
%{Include Only If Paper Has a Subtitle}

\author[SHORT AUTHORS] % (optional, use only with lots of authors)
{
	LONG AUTHOR\inst{1} \and 
	LONG AUTHOR\inst{2}}
% - Give the names in the same order as the appear in the paper.
% - Use the \inst{?} command only if the authors have different
%   affiliation.

\institute[] % (abreviation of university if only one)
{
  \inst{1} UNIVERSITY \and 
  \inst{2} UNIVERSITY % \\ some more info 
}

\date[MFM 2017] % (OPTIONAL, should be abbreviation of conference name)
{ \\[5pt]
Macroeconomic Financial Modeling (MFM) and Macroeconomic Fragility Conference \\[5pt]
PLACE -- DATE}

% - Either use conference name or its abbreviation.
% - Not really informative to the audience, more for people (including
%   yourself) who are reading the slides online

\titlegraphic{\includegraphics[width=2.8cm]{Abbildung/logo}} % single author place in institute

\begin{document}

\begin{frame}
  \titlepage
\end{frame}

\section{Introduction}
%
\subsection{Motivation and Summary}
%
\begin{frame}{some nice title}

beautiful figure 

\end{frame}
%
\begin{frame}{some further title}

Securities Exchange Act had three purposes (Kupiec, 1998)
\begin{itemize}
\item reduction of ``excessive'' credit in securities transactions
\item protection of buyers from too much leverage
\item reduction of stock market volatility
\end{itemize}

\bigskip

Securities Exchange Act of 1934 granted the Federal Reserve Board\\

\bigskip

FRB pursued active margin policy between 1947 and 1974

\end{frame}
%
\begin{frame}{U.S.\ Regulation T}

another figure 

\end{frame}
%
\begin{frame}{Effects of Regulation T}

Kupiec (1998) quote\\

\begin{quote}
{ Margin requirements were ineffective as selective credit controls, inappropriate as rules for investor protection, and were unlikely to be useful in control\-
ling stock price volatility.}
\end{quote}

\bigskip

Fortune (2001)\\

\begin{quote}
{The literature evaluating the effects of Regulation T does provide some evidence that margin requirements affect stock price performance, but the evidence is mixed and it is not clear that the statistical significance found trans\-lates to an economically significant case for an active margin policy.}
\end{quote}

\end{frame}

%
\section{The Economic Model}
%
\subsection{Infinite-horizon Economy}
%
\begin{frame}{Model: Physical Economy}

Infinite-horizon exchange economy in discrete time, $t=0,1,2,\ldots$

\bigskip

Finite number $S$ of i.i.d.\ shocks, $s=1,2,\ldots,S$

\bigskip

History of shocks $s^t =(s_0,s_1,\ldots,s_t)$, called date-event

\bigskip

Single perishable consumption good

\bigskip

$H=2$ types of agents, $h=1,2$, with Epstein-Zin recursive utility

\bigskip

Agent {\color{blue} $h$} receives individual endowment {\color{blue} $e^h(s^t)$} at date-event $s^t$

\end{frame}
%
\section{Margin Requirements and Volatility}
%
\subsection{Basic Observations}
%
\begin{frame}{Collateral Constraints Increase Volatility}

Margin requirement on first asset ${\color{cyan} m_j(s^t)} \equiv 1$, so this asset\\
\mbox{} \hspace{0.25cm} is non-marginable

\bigskip

Aggregated STD of long-lived asset returns: 7.4\%\\
\mbox{} \hspace{0.25cm} (without borrowing: 5.3\%)

\bigskip

Aggregated excess return: 5.0\%

\begin{table}[htdb]
\caption{Asset returns with marginable and non-marginable asset}
\begin{center}
\begin{tabular}{|l|c|c|}
\hline
Asset & STD & ER \\ \hline
Non-marginable ($\delta_1 = 0.04$)    &  8.5     &  6.8 \\ \hline
Marginable ($\delta_2 = 0.07$)       &  7.1     &  4.4  \\ \hline
\end{tabular}
\end{center}
\end{table}

\end{frame}
%

\subsection{Regulation of Margin Requirements}
%
\begin{frame}{Regulating the Stock Market}

Regulation T had small (if any) quantitative impact on stock\\
\mbox{} \hspace{0.25cm}  market volatility (Kupiec 1998, Fortune 2001)

\bigskip

Regulation of ``stock market'' in our model\\[5pt]
\mbox{} \hspace{0.25cm} Asset 1 regulated with constant ${\color{cyan} m_1(s^t)}$\\[5pt]
\mbox{} \hspace{0.25cm} Asset 2 unregulated (endogenous margins)

\bigskip

How does asset return volatility react to changes in ${\color{cyan} m_1}$?

\bigskip

{\color{red} Not much!}

\end{frame}
%
%
\section{Conclusion}
%
\subsection{Summary}
%
\begin{frame}{Assumptions and Limitations}

General equilibrium model ignores institutional details

\bigskip

Technical limitations require
\begin{itemize}
\item short-sale constraints on long-lived assets
\item two types of agents
\end{itemize}

\bigskip

Countercyclical margins depend on exogenous shock but\\
\mbox{} \hspace{0.25cm} should depend on price levels instead

\bigskip

\pause

\bigskip

Model provides insights into {\color{red} general equilibrium effects}\\
\mbox{} \hspace{0.25cm} of {\color{red} margin regulation}

\end{frame}
%
\begin{frame}{Summary}

Calibrated general equilibrium infinite-horizon economy with\\
\mbox{} \hspace{0.25cm} heterogeneous agents and collateral constraints

\begin{itemize}
\item Collateralized borrowing increases return volatility of long-lived assets

\item Changes of margin requirements (as under Regulation T) have little effect if other long-lived assets are not regulated

\item Spillover effects: If margins on one asset are increased, the volatility of other assets decreases

\item Changes of margin requirements may have strong effects when all markets are regulated
\end{itemize}

\end{frame}
%
\end{document}
